\documentclass[12pt,a4paper]{article}
\title{MATH1110 Lec 009 Week 5-M}
\author{Benjamin Thompson}
\date{September 20, 2021}

\newcommand{\vspaceC}{\vspace{2cm}}

\usepackage{amsmath}
\usepackage{graphicx}

\usepackage{fancyhdr}
\pagestyle{fancy}

\fancyhf{}
\lhead{MATH1110 Sec 009}
\chead{Week 5-M}
\rhead{September 20, 2021}
\cfoot{\thepage}

\newtheorem{theorem}{Example}


\begin{document}
\subsection*{Proving functions are equal or different}
In mathematics, two functions are considered equal if they have the same domain and take the same value at each point in the domain.
\\
\\
\textbf{Example:} Show that $f(x) = 2x$ and $g(x) = 3x$ are different functions.
\\
\\
\emph{Proof:} Since $f(1) = 2 \ne 3 = g(1)$, $f$ and $g$ take different values at $1$. Therefore $f \ne g$.
\\
\\
Sometimes it is not as obvious when two functions are equal:
\\
\\
\textbf{Example:} Let
\[
    f(x) = \frac{x^3 - x}{x + 1}, \qquad g(x) = x^2 - x
\]
have domain positive real numbers. Is $f = g$?
\\
\\
\emph{Solution:} 
Note
\[
    f(x) = \frac{x^3 - x}{x+1} = \frac{x(x^2 - 1)}{x+1} = \frac{x(x+1)(x-1)}{x+1} = x(x-1) = x^2 - x = g(x).
\]

\section*{Problems}
Are the following functions equal? Justify your answers. In particular, if the functions are not equal, prove this by finding a point in the domain where the functions take different values.

\begin{enumerate}
    \item $f(x) = x^2 - 2x - 8, \qquad g(x) = (x+4)(x-2)$.
    \vspaceC
    \item $f(x) = x^2 - 2x - 3, \qquad g(x) =  (x+1)(x-3)$.
    \vspaceC
    \item $f(x) = x^2 + 1, \qquad g(x) = (x + 1)^2$.
    \vspaceC
    \item $f(x) = x^3 - 8, \qquad g(x) =  (x-2)^3$.
    \vspaceC
    \item $f(x) = \sqrt[3]{x^3 + 3x^2 + 3x + 1}, \qquad g(x) = x + 1$.
    \vspaceC
    \item $f(x) = \sqrt[3]{x^3 + 1}, \qquad g(x) = \sqrt[3]{x^3} + \sqrt[3]{1}$.
    \vspaceC
    \item $f(x) = \sin 2x, \qquad g(x) = 2\sin x$.
    \vspaceC
    \item
       \[ f(x) = x^3 - x^2 - x + 1, \qquad g(x) = \begin{cases} 
          (x^4 - 2x^2 + 1)/(x+1) & x \ne 0 \\
          0 & x = -1.
       \end{cases}
    \]
        \vspaceC
    \item
       \[ f(x) = x^2 + 2, \qquad g(x) = \begin{cases} 
          (x^3 - 1)/(x-1) & x \ne 1 \\
          3 & x = 1
       \end{cases}.
    \]
\vspaceC
       \[ f(x) = x^4 + x^3 + x^2 + x + 1, \qquad g(x) = \begin{cases} 
          (x^5 - 1)/(x-1) & x \ne 1 \\
          5 & x = 1
       \end{cases}.
    \]
    \vspaceC
    \item $f(x,y) = x^2 - y^2, \qquad g(x,y) = (x-y)(x+y).$
    \vspaceC
    \item $f(x,y) = (x-y)^2(x+y)^2, \qquad g(x,y) = x^4 - y^4.$
            \vspaceC
    \item $f(x,y) = 10^x + 10^y, \qquad g(x,y) = 10^{x + y}$.
        \vspaceC
    \item $f(x,y) = \frac{1}{x+y}, \qquad g(x,y) = \frac{1}{x} + \frac{1}{y}.$
        \vspaceC
    \item $f(x,y) = \cos(x+y), \qquad g(x,y) = \cos x + \cos y.$
    \vspaceC
    \item $f(x,y,z) = (x + y)^z, \qquad g(x,y,z) = x^z + y^z.$
\end{enumerate}


\end{document}
