\documentclass[12pt, a4paper]{article}

\title{MATH2210 Fall 2023 Problem Sheet Week 2}
\author{Benjamin G. Thompson}
\date{August 30, 2023}

\usepackage[left=2cm, right=2cm, top=2cm, bottom=2cm]{geometry}
\usepackage{fancyhdr}
\pagestyle{fancy}

\fancyhf{}
\lhead{MATH2210}
\chead{Week 2-W}
\rhead{August 30, 2023}
\cfoot{}

\begin{document}
\section*{First part}
\begin{itemize}
\item Given a line, a circle, and a point on the circle, the line is said to be \emph{tangent to the circle at the point} if it intersects the circle at the point and nowhere else. Such a line is called a \emph{tangent line} to the circle.
\item Let $S^1$ refer to the circle in the Euclidean plane centered at the origin with radius 1.
\end{itemize}
\textbf{Main question:}
Characterize all tangent lines to $S^1$.
\begin{itemize}
  \item Specifically, any point on $S^1$ can be described in terms of an angle $\theta$.\footnote{E.g. The right-most point of $S^1$ corresponds to $\theta = 0$, the top-most point of $S^1$ corresponds to $\theta = \pi / 2$.} What is an equation describing the tangent line to $S^1$ at the point corresponding to $\theta$?
\end{itemize}

\newpage
\section*{Second part}
\begin{enumerate}
\item Characterize all matrices in row-echelon form with the following property: If any entry is changed, the matrix is no longer in row-echelon form.
\item Create a system of three linear equations in three variables whose solution set geometrically corresponds to:
  \begin{enumerate}
  \item The empty set
  \item A point
  \item A line
  \item A plane
  \item Euclidean 3-space.
  \end{enumerate}
  \item In the example above whose solution set is a plane, describe the solution set in terms of a linear combination of vectors.
\end{enumerate}
\end{document}
